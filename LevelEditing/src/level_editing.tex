\documentclass{article}
\usepackage{amsmath}
\usepackage{amssymb}
\usepackage{bm}
\usepackage{cancel}
\usepackage{hyperref}
\usepackage{tikz}

\title{Styx and Stones: Level Editing Flow and Architecture}
\author{
    \large
    Devin McGinty\\
    \normalsize
    dlm348@drexel.edu
}

\begin{document}
\maketitle

\tableofcontents
\newpage

%%%%%%%%%%%%%%%%%%%%%%%%%%%%%%%%%%%%%%%%%%%%%%%%%%%%%%%%%%%%%%%%%%%%%%%%%%%%%%%%%%%%%%%%%%%%%%%%%%%%%%%%%%%%%%%%%%%%%%%
\section{Overview}
Styx and Stones is a 2D platforming game created by the FoodTaste Drexel Senior Project Team consisting of members
\begin{itemize}
    \item \href{mailto:jinu.jacob@drexel.edu}{Jinu Jacob}
    \item \href{mailto:brendan.k.kelley@drexel.edu}{Brendan Kelley}
    \item \href{mailto:bailey.s.myers-morgan@drexel.edu}{Bailey Myers-Morgan}
    \item \href{mailto:dlm348@drexel.edu}{Devin McGinty}
    \item \href{mailto:le.n.nguyen@drexel.edu}{Le Nguyen}
    \item \href{mailto:nicole.m.vecere@drexel.edu}{Nicole Vecere}
\end{itemize}

The purpose of this document is to demonstrate the process of designing levels and loading them into an instance of
the \textsc{Styx and Stones} software program for the user to interact with. In the rest of this document the term
``the game'' refers to the ``\textsc{Styx and Stones}'' game program. The general process is to design the level in the
\href{http://www.mapeditor.org}{\textsc{Tiled} 2D map editor}, make the level file readable in Unity using the
\href{http://www.seanba.com/tiled2unity}{\textsc{Tiled2Unity}} command line tool, and read the file using a C#
deserializer written in Unity.

%%%%%%%%%%%%%%%%%%%%%%%%%%%%%%%%%%%%%%%%%%%%%%%%%%%%%%%%%%%%%%%%%%%%%%%%%%%%%%%%%%%%%%%%%%%%%%%%%%%%%%%%%%%%%%%%%%%%%%%
\section{Architecture}
\subsection{Tiled Map Editor}
\subsection{Tiled2Unity Deserialization}
\subsection{Unity}

%%%%%%%%%%%%%%%%%%%%%%%%%%%%%%%%%%%%%%%%%%%%%%%%%%%%%%%%%%%%%%%%%%%%%%%%%%%%%%%%%%%%%%%%%%%%%%%%%%%%%%%%%%%%%%%%%%%%%%%
\section{Loading Levels}

%%%%%%%%%%%%%%%%%%%%%%%%%%%%%%%%%%%%%%%%%%%%%%%%%%%%%%%%%%%%%%%%%%%%%%%%%%%%%%%%%%%%%%%%%%%%%%%%%%%%%%%%%%%%%%%%%%%%%%%
\section{Current Challenges}

%%%%%%%%%%%%%%%%%%%%%%%%%%%%%%%%%%%%%%%%%%%%%%%%%%%%%%%%%%%%%%%%%%%%%%%%%%%%%%%%%%%%%%%%%%%%%%%%%%%%%%%%%%%%%%%%%%%%%%%
\end{document}

